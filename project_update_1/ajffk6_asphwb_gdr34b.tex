\documentclass[12pt,a4paper]{report}
\setcounter{secnumdepth}{0}
\usepackage{titlesec}
\titleformat{\chapter}[block]{\Large\bfseries\filcenter}{}{1em}{}
\titleformat{\section}[block]{\Large\bfseries\filcenter}{}{1em}{}
\titleformat{\subsection}[block]{\Large\bfseries\filcenter}{}{1em}{}
\usepackage{amsmath}
\usepackage{helvet}
\renewcommand{\familydefault}{\sfdefault}
\usepackage[pdftex]{graphicx}
\usepackage[table]{xcolor}

\author{
    Andrew Fallgren \\
    \texttt{ajffk6@mst.edu}
    \and
    Aaron Pope \\
    \texttt{aaron.pope@mst.edu}
    \and
    George Rush \\
    \texttt{gdr34b@mst.edu}
}
\title{Defending Against Wormhole Attacks Using AODV Routing}
\begin{document}
\maketitle

\pagebreak
\section{Introduction}
In this work, we suggest using a modified version of the AODV routing protocol to defend against wormhole attacks in ad-hoc wireless networks. Ad-hoc On-Demand Distance Vector Routing (AODV) is a routing algorithm for ad-hoc networks in which routes are obtained only as needed. The algorithm's primary objectives are to broadcast discovery packets only when necessary, to distinguish between neighborhood detection and general topology maintenance, and to disseminate information about changes in local connectivity to those neighbors that are likely to need the information~\cite{749281}. In general, AODV will always select the shortest route to its destination for any given network transaction, but routing packets can be misused to enable a variety of attacks such as route disruption, node isolation or resource consumption~\cite{Ning2005795}.

In a wormhole attack, an attacker sniffs packets at one point in the network, tunnels them elsewhere in the network, and replays them. By using a long-range directional antenna, an attacker can make certain nodes appear to have a shorter hop distance to other parts of the network than any of their neighbors. As such, this type of attack is especially effective against AODV since an attacker can easily become part of the shortest route between two points in the network~\cite{1589115}. Furthermore, because it is unnecessary to compromise nodes in the network, all nodes continue to respond as expected by the protocol. This makes attack detection a difficult task.

A black hole attack is an extension on the wormhole attack. To execute it, a wormhole is used to establish routes between nodes, and then all data packets on those routes are silently dropped. In this way, an attacker can launch a permanent denial-of-service attack.

\section{Related Work}
A variety of methods have been used to improve the overall security of networks employing AODV routing. Some techniques introduce authentication to the message passing within the network~\cite{Hu:2005:ASO:1160100.1160103, 1181388, 806983}. While this approach is very powerful, it can dramatically increase the computational load on participating nodes and in the case of asymmetric encryption, can require a centralized source for public keys which may not be feasible for some wireless network applications.

Alternative approaches use intrusion detection systems to discover malicious node behavior. Statistical methods can be used to distinguish malicious packet dropping from packet dropping that occurs as a result of network congestion~\cite{1258776}. The requirement for predicting congestion limits the usefulness of this approach when dealing with networks which exhibit sporadic and bursty traffic. Other attempts at constructing intrusion detection systems for ad-hoc networks rely on the strategic placement of additional hardware to monitor network traffic for malicious behavior~\cite{Tseng:2003:SID:986858.986876}. While the additional hardware requirements might be feasible for some applications where security is of the utmost importance, this approach is impractical for most lightweight ad-hoc networks.

\section{Our Approach}
In our technique, we consider that source and destination nodes can exchange the number of sent and received data packets when establishing routes through the packets sent out in the AODV protocol. Since AODV floods the network when establishing routes, a polling of the neighbors can be used to detect cases where a node is silently dropping packets or if the shortest is not a reliable route.  It could then be possible for alternative routes of similar length to be chosen instead. This general purpose approach requires a minimal alteration to the AODV protocol and does not require computationally expensive encryption, a centralized authentication source or any additional monitoring hardware.

\section{Evaluation}
In order to compare our solution to other techniques for preventing or countering wormhole attacks, we propose building a network simulation and testing various attack scenarios. That simulation, relevant scenarios, and the evaluation metric are discussed below.

\subsection{Network Simulation}
The network would be represented as a graph, with edges connecting nodes in communication range. Random nodes would be chosen to form routes using the AODV protocol, and dummy messages would then be sent one or both ways to simulate data transactions. Periodically, routes would be refreshed as per AODV's rules. Attackers attempting to execute black hole attacks would do so depending on scenarios defined in the next section. The simulation would end either after a fixed period of time or when a fixed number of messages had been sent across the network.

\subsection{Attack Scenarios}
Several scenarios would be beneficial for comparison:
\begin{itemize}
    \item No black holes occur. This would help test how proposed solutions affect normal network operation.
    \item A single black hole is formed. This would show how proposed solutions work under ideal conditions.
    \item Multiple black holes are formed. This would test what happens when the enemy has greater capabilities and can further disrupt the network.
\end{itemize}
Which scenarios are most relevant depends heavily on the enemy's capabilities and the type of network being deployed. Each run of the simulation would return the evaluation metric covered in the next section.

\subsection{Evaluation Metric}
Regarding black hole attacks, we take into account two main concerns for a network operator: the number of lost messages and the message complexity of any proposed solution. Suppose we denote the number of lost messages as $L$ and the message complexity as $M$. Minimizing $L$ increases availability, and minimizing $M$ decreases power usage. However, depending on the purpose of any given network and the capabilities of its nodes, one may prioritize either availability or power usage. Rather than choosing specific weights for the end user, we would choose instead to view this as a multi-objective problem. This would allow the end user to decide the relative weight of the two metrics themselves, adapting the evaluation to their particular needs.

\bibliographystyle{abbrv}
\bibliography{references}

\end{document}
